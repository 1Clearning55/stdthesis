% !TeX root = ../Template.tex
% 本LaTeX模板的使用说明

\chapter{如何使用STDsthesis模板}
\section{样例项目}
我对教务处提供的Word模板进行了适配,大致是符合了学校的要求。如果你在使用中出现了问题,希望你能联系我,邮箱:\href{uquantum@hotmail.com}{uquantum@hotmail.com},我会及时修正。目前本文档可以直接使用或者用于参考学习:
出于性能和管理方面的考虑,stdthesis使用分布式的源文件方案,将论文的各个部分(通常以章为单位)分散到tex文件中,然后在主文档Template.tex中统一处理。如下展示了一个可能的文件目录情况。
\dirtree{%
		.1 \myfolder{pink}{工作文件夹}.
		.2 \myfolder{cyan}{stdthesis.cls}.
		.2 \myfolder{cyan}{STDthesisextra.cls}.
		.2 \myfolder{cyan}{Template.tex}.
		.2 \myfolder{cyan}{chapter}.
		.3 \myfolder{lime}.{tex/chapintro}.
		.3 \myfolder{lime}.{tex/chapinstruction}.
		.3 \myfolder{lime}.{tex/chapbegin}.
		.3 \myfolder{lime}.{tex/chapsample}.
		.3 \myfolder{lime}.{tex/chapsummary}.
		.3 \myfolder{lime}.{tex/chapappendix}.
		.3 \myfolder{lime}.{tex/chapacknowledge}.
		.3 \myfolder{lime}.{tex/chapbiography}.
		.2 \myfolder{cyan}{figures}.
		.3 \myfolder{lime}{logo-std.jpg}.
		.3 \myfolder{lime}{sample.jpg}.
		.2 \myfolder{cyan}{reference}.
		.3 \myfolder{lime}{reference.bib}.
	}%\dirtree

此处为公式演示:
\begin{equation}
\alpha=\dfrac{lnp_2-lnp_1}{t_2-t_1}\label{eq:zuning}
\end{equation}
\section{构建文档}
xeCJK 是提供 LaTeX 中文支持的宏包,并且依赖于 XeLaTeX,因此,我们需要使用 xelatex 命令进行构建。
LaTeX 在构建交叉索引时需要多次运行,才能最终解析所有的引用,并且期间需要 BibTeX 对参考文献数据库进行处理。因此,一般的手动构建命令是:

1.xelatex main

2.bibtex main

3.xelatex main

4.xelatex main

或者强烈建议采用图形化的编译器Texstudio\footnote{下载地址:\href{http://texstudio.sourceforge.net/}{http://texstudio.sourceforge.net/}}进行编译,
