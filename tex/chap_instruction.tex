% !TeX root = ../Template.tex
% 本LaTeX模板的一般使用说明
\chapter{说明}

Again,这是铁大论文\LaTeX{}模板(\CTeX{}-Based)\STDThesis{}。石家庄铁道大学论文模板的一些说明

本\LaTeX{}模板为铁大本科生学位论文模板,适用于理工类博士、学术硕士和专业硕士。本\LaTeX{}模板参考自教务处下发的《本科生毕业论文手册》,具体要求请参见各自的《手册》,最终成文格式需参考学院要求及打印方意见。本模板中大量内容和说明直接摘抄自《手册》(2015年8月版),基本覆盖了论文内容和格式方面的要求。

本模板已上传\href{https://github.com/dancvv/stdthesis}{GitHub}。
%-----------------------------
\section{宏包使用}

请将以下文件与此LaTeX文件放在同一目录中:

\begin{tabular}{ll}
 \verb|STD.cls |             & $\triangleright$ LaTeX宏模板文件 \\
 \verb|STD_mac.cls |         & $\triangleright$ LaTeX宏模板文件(For Mac with XeLaTeX) \\
 \verb|GBT7714-2005.bst|      & $\triangleright$ 国标参考文献BibTeX样式文件2005 \\
 \verb|GBT7714-2015.bst|      & $\triangleright$ 国标参考文献BibTeX样式文件2015 \\
 \verb|tex/*.tex|             & $\triangleright$ 本模板样例中的独立章节\\
\end{tabular}\\

通过 \verb|\documentclass[<thesis>,<permission>,<printtype>,<ctexbookoptions>]{STD}|载入宏包:
\begin{itemize}[leftmargin=3cm]
  \item[{\tt thesis} $\triangleright$]  论文类型(thesis),可选值:\\
    a) 本科毕业论文(\verb|master|)[缺省值]\\
    b) 专业硕士论文(\verb|professional|)\\
    c) 博士论文(\verb|doctor|)
  \item[{\tt printtype} $\triangleright$] 打印属性(printtype),可选值: \\
    a) 单面打印(\verb|onside|)[缺省值]\\
    b) 双面打印(\verb|twoside|)
  \item[{\tt ctexbookoptions} $\triangleright$] {\tt ctexbook}文档类支持的其他选项: \\
    使用{\tt ctexbookoptions}选项传递{\tt ctexbook}文档类支持的其他选项。
    例如,使用{\tt fontset=founder}选项启用方正字体以避免生僻字乱码的问题\footnote{需要系统安装方正字体。}。
\end{itemize}

模板已内嵌LaTeX工具包:
 {\tt ifthen},{\tt etoolbox},{\tt titletoc},{\tt remreset},{\tt remreset},
 {\tt geometry},{\tt fancyhdr},{\tt setspace},{\tt caption},{\tt float},
 {\tt graphicx},{\tt subfigure},{\tt epstopdf},
 {\tt book\-tabs},{\tt longtable},{\tt multirow},{\tt array}, {\tt enumitem},
 {\tt algorithm2e},{\tt amsmath},{\tt amsthm},{\tt listings},
 {\tt pifont},{\tt color},{\tt soul}, {\tt newtxtext}, {\tt newtxmath}。

模板已内嵌宏:\verb|\highlight{text}|(黄色高亮)。

请统一使用UTF-8编码。



%-----------------------------
\section{选项设置}

\begin{itemize}[leftmargin=3cm]
  \item[{\tt  $\backslash$refcolor} $\triangleright$]  开启/关闭引用编号颜色,包括参考文献,公式,图,表,算法等\\
  \texttt{on}:开启 [默认]\\
  \texttt{off}:关闭
  \item[{\tt $\backslash$beginright} $\triangleright$]  摘要和正文从右侧开始\\
  \texttt{on}:开启 [默认]\\
  \texttt{off}:关闭
  \item[{\tt $\backslash$emptypageword} $\triangleright$]  空白页留字
  \item[{\tt $\backslash$Listfigtab} $\triangleright$]  是否使用图标清单目录\\
  \texttt{on}:开启 [默认]\\
  \texttt{off}:关闭
\end{itemize}


%-----------------------------
\section{章节撰写}
本模板支持以下标题级别标题级别:

\begin{tabular}{ll}
  \verb|\chapter{章}|              & $\triangleright$ 第一章 \\
  \verb|\chapter*{无章号章}|       & $\triangleright$ 无章号章 \\
  \verb|\chaptera{无章号有目录章}| & $\triangleright$ 无章号有目录章 \\
  \verb|\summary|                  & $\triangleright$ 总结\\
  \verb|\appendix|                 & $\triangleright$ 附录\\
  \verb|\achievement|              & $\triangleright$ 攻读学位期间取得的成果\\
  \verb|\acknowledgments|          & $\triangleright$ 致谢\\
  \verb|\biography|                & $\triangleright$ 作者简介\\
  \verb|\section{节}|              & $\triangleright$ 1.1 节\\
  \verb|\subsection{条}|           & $\triangleright$ 1.1.1 条\\
  \verb|\subsubsection{A}|         & $\triangleright$ 1.1.1.1 A\\
  \verb|\paragraph{a}|             & $\triangleright$ 1.1.1.1.1 a\\
  \verb|\subparagraph{a)}|         & $\triangleright$ 1.1.1.1.1.1 a)\\
\end{tabular}

%-----------------------------
\section{注意事项}
\begin{itemize}
  \item[$\triangleright$] \textit{中文斜体}将转换为楷体;
  \item[$\triangleright$] \verb|STD.cls|采用包{\tt newtxtext}和{\tt newtxmath},\textbf{中文粗体}在Windows下转换为黑体(有可能是因为newtx包没安装好,By WeiQM),Linux下正常(By QiaoJF);
  \item[$\triangleright$] \verb|STD_mac.cls|采用包{\tt times},\textbf{中文粗体}转换为黑体(By CaiBW);
  \item[$\triangleright$] \verb|\label{<text>}|中不能使用中文;
  \item[$\triangleright$] 浮动体与正文之间的距离是弹性的;
  \item[$\triangleright$] 命令符与汉字之间请注意加空格以避免undefined错误(pdfLaTeX下好像一般不存在这个问题,主要在XeLaTeX编译环境下发生);
\end{itemize}

%-----------------------------
\section{ToDo}
\begin{itemize}
  \item[$\triangleright$] 数学环境的行间隔;
  \item[$\triangleright$] 参考文献的行间隔;
\end{itemize}

%-----------------------------
\section{意见及问题反馈}

\indent E-mail:

\indent GitHub:\href{https://github.com/CheckBoxStudio/STDThesis/issues}{https://github.com/CheckBoxStudio/STDThesis/issues}
